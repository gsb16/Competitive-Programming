Está chegando a grande final do Campeonato Nlogonense de Surf Aquático.
Ano que vem, a final ocorrerá na cidade de Foça do Iguachim (FdI)!
A cidade é famosa por conter o Parque Nacional do Iguachim, que conta com várias
atrações. Dentre elas, destacam-se as Cataratas do Iguachim, um dos pontos turisticos
mais famosos de Nlogonia!

O Parque conta com $N$ atrações, numeradas de $1$ a $N$. As atrações são
dispostas em uma linha reta no Parque. Desta forma, o Parque pode ser
descrito como uma rua contendo entradas para as atrações $1$, $2$, ..., $N$, onde a
atração $1$ é a mais próxima da entrada do Parque, enquanto a atração $N$
é a mais próxima da saída do Parque. Para não tumultoar o Parque, é
exigido que as atrações sejam visitadas \textit{em ordem} da entrada para a
saída, isto é, se você visitar a atração $i$, você não pode voltar e visitar as
atrações $1$, $2$, ..., $i-1$.

Além disso, existem dois tipos de \textit{tickets} no Parque: os
\textit{tickets} verdes e os \textit{tickets} amarelos. Cada uma das $N$
atrações exigem, como pagamento por sua entrada, uma certa quantidade de
\textit{tickets} de exatamente um tipo. Ao entrar em uma atração, o Parque pode
lhe presentear com uma certa quantidade de \textit{tickets} do outro tipo, isto
é, uma atração que cobra \textit{tickets} verdes como entrada pode lhe dar
\textit{tickets} amarelos como presente, ou vice-versa. Você pode não poder
entrar em uma atração se não tiver \textit{tickets} suficientes para ela, mas
também pode optar não entrar nela se quiser, mesmo se tiver \textit{tickets}
suficientes.

Entretanto, você quer aproveitar o Parque o máximo possível! Dada a quantidade
inicial de \textit{tickets} de cada tipo que você possui e a descrição das atrações do
Parque, determine o número máximo de atrações que podem ser visitadas.

\subsection*{Entrada}

A primeira linha contém o inteiro $N$ ($1 \leq N \leq 40$). A segunda linha
contém dois inteiros $V$ e $A$ ($0 \leq V, A \leq 20$), o número de
\textit{tickets} verdes e amarelos que você possui inicialmente. As próximas $N$
linhas descrevem as atrações do Parque, na ordem da entrada para a saída do
Parque.
Cada linha contém dois inteiros $V_i$ e $A_i$ ($-20 \leq V_i, A_i \leq 20$, $V_i
\times A_i < 0$). Se $V_i < 0$, a atração cobra $|V_i|$ \textit{tickets}
verdes como entrada, e, se visitada, lhe presenteia com $A_i$ \textit{tickets}
amarelos. Caso contrário, ela cobra $|A_i|$ \textit{tickets} amarelos, e, se
visitada, lhe presenteia com $V_i$ \textit{tickets} verdes.

\subsection*{Saída}

Imprima uma linha contendo a quantidade máxima de atrações que podem ser
visitadas.

\begin{table}[!h]
\centering
\begin{tabular}{|l|l|}
\hline
\begin{minipage}[t]{3in}
\textbf{Exemplo de entrada}
\begin{verbatim}
3
10 0
-10 5
-5 4
20 -5
\end{verbatim}
\vspace{1mm}
\end{minipage}
&

\begin{minipage}[t]{3in}
\textbf{Exemplo de saída}
\begin{verbatim}
2
\end{verbatim}
\vspace{1mm}
\end{minipage} \\
\hline
\end{tabular}
\end{table}
