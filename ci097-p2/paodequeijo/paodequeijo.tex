Está chegando a grande final do Campeonato Nlogonense de Surf Aquático, que este
ano ocorrerá na cidade de Bonita Horeleninha (BH)! Nesta cidade, o jogo
\textit{PãodequeijoSweeper} é bastante popular!

O tabuleiro do jogo consiste em uma matriz de $N$ linhas e $M$ colunas. Cada
célula da matriz contém um pão de queijo ou o número de pães de queijo que
existem nas celulas adjacentes a ela. Uma célula é adjacente a outra se estiver
imediatamente à esquerda, à direita, acima ou abaixo da célula. Note que, se não
contiver um pão de queijo, uma célula deve obrigatoriamente conter um número
entre $0$ e $4$, inclusive.

Dadas as posições dos pães de queijo, determine o tabuleiro do jogo!

\subsection*{Entrada}

A primeira linha contém os inteiros $N$ e $M$ ($1 \leq N, M \leq 100$). As
próximas $N$ linhas contém $M$ inteiros cada, separados por espaços, descrevendo
os pães de queijo no tabuleiro. O $j-$ésimo inteiro da $i-$ésima linha é $1$ se
existe um pão de queijo na linha $i$ e coluna $j$ do tabuleiro, ou $0$ caso
contrário.

\subsection*{Saída}

Imprima $N$ linhas com $M$ inteiros cada, \textit{não separados por espaços},
descrevendo a configuração do tabuleiro. Se uma posição contém um pão de queijo,
imprima \verb|9| para ela; caso contrário, imprima o número cuja posição deve
conter.

\begin{table}[!h]
\centering
\begin{tabular}{|l|l|}
\hline
\begin{minipage}[t]{3in}
\textbf{Exemplo de entrada}
\begin{verbatim}
4 4
0 0 1 1
0 1 0 1
0 0 1 0
1 1 0 1
\end{verbatim}
\vspace{1mm}
\end{minipage}
&

\begin{minipage}[t]{3in}
\textbf{Exemplo de saída}
\begin{verbatim}
0299
1949
1393
9939
\end{verbatim}
\vspace{1mm}
\end{minipage} \\
\hline
\end{tabular}
\end{table}

\begin{table}[!h]
\centering
\begin{tabular}{|l|l|}
\hline
\begin{minipage}[t]{3in}
\textbf{Exemplo de entrada}
\begin{verbatim}
1 2
0 1
\end{verbatim}
\vspace{1mm}
\end{minipage}
&

\begin{minipage}[t]{3in}
\textbf{Exemplo de saída}
\begin{verbatim}
19
\end{verbatim}
\vspace{1mm}
\end{minipage} \\
\hline
\end{tabular}
\end{table}
