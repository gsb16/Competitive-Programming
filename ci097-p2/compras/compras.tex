Está chegando a grande final do Campeonato Nlogonense de Surf Aquático.
Ano que vem, a final ocorrerá na cidade de Foça do Iguachim (FdI)!
A região de FdI e das cidades próximas é famosa por seu comércio, composto por diversas lojas
que costumam vender diversos produtos a preços mais atraentes que no restante do
país. Você quer aproveitar a viagem para FdI para comprar o novo celular
\textit{Aifôni (R)}!\footnote{Na verdade, você queria um \textit{Sãosunga (R)},
mas este celular é um verdadeiro estouro!}

Existem $N$ lojas na região, numeradas de $1$ a $N$. Todas as lojas vendem o
celular, embora o preço do aparelho pode ser diferente em cada loja. Para não
tornar sua viagem cansativa, você pode considerar não visitar todas as $N$
lojas, mas sim visitar apenas as lojas entre duas dadas lojas $i$ e $j$, inclusive.
Você está interessado na \textit{maior diferença de preços} do aparelho entre as
lojas visitadas. A diferença é dada por $|M - m|$, onde $M$ é o maior preço
dentre as lojas visitadas, e $m$ é o menor.

Além disso, as lojas podem alterar o preço do celular como desejarem! Sua tarefa
é determinar, \textit{para várias consultas}, a maior diferença de preços nas
lojas entre duas dadas lojas, considerando também eventuais alterações de preços nas lojas.

\subsection*{Entrada}

A primeira linha contém o inteiro $N$ ($1 \leq N \leq 10^5$). A segunda linha
contém $N$ inteiros $p_1, p_2, ..., p_N$ ($1 \leq p_i \leq 10^5$). O inteiro
$p_i$ indica o preço inicial do celular na loja $i$.
A terceira linha contém um inteiro $Q$ ($1 \leq Q \leq 10^5$), o número de
operações. As próximas $Q$ linhas descrevem uma operação cada. Cada operação
pode ser descrita de duas formas:

\begin{itemize}
    \item $1$ $i$ $p$ ($1 \leq i \leq N, 1 \leq p \leq 10^5$),
    indicando que o preço do celular foi alterado para $p$ na loja $i$.
    \item $2$ $i$ $j$ ($1 \leq i \leq j \leq N$), indicando uma consulta.
\end{itemize}

\subsection*{Saída}

Para cada consulta, imprima uma linha contendo a maior diferença de preços das
lojas entre as lojas $i$ e $j$, inclusive.

\begin{table}[!h]
\centering
\begin{tabular}{|l|l|}
\hline
\begin{minipage}[t]{3in}
\textbf{Exemplo de entrada}
\begin{verbatim}
4
100 150 90 170
3
2 1 3
1 2 50
2 2 4
\end{verbatim}
\vspace{1mm}
\end{minipage}
&

\begin{minipage}[t]{3in}
\textbf{Exemplo de saída}
\begin{verbatim}
60
120
\end{verbatim}
\vspace{1mm}
\end{minipage} \\
\hline
\end{tabular}
\end{table}
