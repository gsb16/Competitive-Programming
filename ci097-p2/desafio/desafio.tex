O \textit{Desafio das Moedas Prateadas} é um esporte individual popular no reino de Diddykongolândia. O campo e as regras do jogo são descritos a seguir.

O campo do jogo consiste em \textit{locais} e \textit{trechos}. Há $N$ locais no
campo. Um desses locais é o local de \textit{largada}, no qual o jogador inicia o jogo.

Há $M$ trechos no campo. Cada trecho é uma via unidirecional que liga um local do campo a outro local distinto. Os trechos são os únicos meios pelos quais o jogador pode se mover entre os locais do campo. Os trechos do campo são dados de tal forma que:

\begin{itemize}
    \item é possível ir do local de largada a qualquer outro local usando os trechos;
    \item é possível ir de qualquer local do campo ao local de largada usando os trechos;
    \item é impossível sair de um local $l$ e voltar para o mesmo local $l$ sem passar pelo local de largada.
\end{itemize}

Há também $K$ \textit{moedas prateadas} no jogo. Cada moeda está em um local distinto do campo. Se o jogador chegar a um local que contém uma moeda, o jogador pode coletá-la. Não há moeda no local de largada.

O jogador começa o jogo no local de largada. Uma \textit{volta} é completada pelo jogador quando ele retorna ao local de largada após passar por outro(s) local(is).

O jogador deve completar exatamente três voltas. Se o jogador conseguir coletar todas as moedas de prata antes de completar a última volta, ele vence. Caso contrário, ele perde.

Após analizar o campo de jogo, você descobriu quanto tempo leva para atravessar cada trecho. Agora, você deve descobrir se é possível vencer no campo dado e, em caso positivo, qual o tempo mínimo que você deve levar para vencer. Considere instantâneo o tempo para coletar uma moeda e para atravessar um local (sair de um trecho e entrar em outro adjacente).


\subsection*{Entrada}


A entrada inicia com uma linha contendo três inteiros $N$, $M$ e $K$ ($2 \leq N
        \leq 1000, 2 \leq M \leq (N^2 + N - 2)/2, 1 \leq K \leq min\{12, N-1\}$), indicando o número de locais, de trechos e de moedas no campo. Os locais são numerados de 1 a $N$. O local 1 é o local de largada.

As próximas $M$ linhas descrevem os trechos. Cada trecho é descrito por três
inteiros $l_a$, $l_b$ e $t$ ($1 \leq l_a, l_b \leq N, l_a \neq l_b, 1 \leq t
        \leq 10^4$), indicando que há um trecho que leva do local $l_a$ para o
local $l_b$ que é atravessado em $t$ segundos.

A última linha contém $K$ inteiros distintos $k_i$ ($2 \leq k_i \leq N$ para $1 \leq i \leq K$) indicando os locais das moedas prateadas.

\subsection*{Saída}

Se não é possível vencer o jogo, imprima uma linha contendo
``\verb|impossivel|'' (sem aspas). Caso
contrário, imprima uma linha contendo o menor tempo necessário, em segundos, para
vencer o jogo.

\clearpage
\begin{table}[!h]
\centering
\begin{tabular}{|l|l|}
\hline
\begin{minipage}[t]{3in}
\textbf{Exemplo de entrada}
\begin{verbatim}
4 6 3
1 2 1
1 3 1
1 4 1
2 1 1
3 1 1
4 1 1
2 3 4
\end{verbatim}
\vspace{1mm}
\end{minipage}
&

\begin{minipage}[t]{3in}
\textbf{Exemplo de saída}
\begin{verbatim}
6
\end{verbatim}
\vspace{1mm}
\end{minipage} \\
\hline
\end{tabular}
\end{table}

\begin{table}[!h]
\centering
\begin{tabular}{|l|l|}
\hline
\begin{minipage}[t]{3in}
\textbf{Exemplo de entrada}
\begin{verbatim}
5 7 2
1 2 1
1 3 2
2 3 3
3 4 7
3 5 3
4 5 2
5 1 4
2 4
\end{verbatim}
\vspace{1mm}
\end{minipage}
&

\begin{minipage}[t]{3in}
\textbf{Exemplo de saída}
\begin{verbatim}
35
\end{verbatim}
\vspace{1mm}
\end{minipage} \\
\hline
\end{tabular}
\end{table}

\begin{table}[!h]
\centering
\begin{tabular}{|l|l|}
\hline
\begin{minipage}[t]{3in}
\textbf{Exemplo de entrada}
\begin{verbatim}
5 8 4
1 2 3
1 3 2
1 4 10
1 5 7
2 1 5
3 1 3
4 1 1
5 1 12
4 5 3 2
\end{verbatim}
\vspace{1mm}
\end{minipage}
&

\begin{minipage}[t]{3in}
\textbf{Exemplo de saída}
\begin{verbatim}
impossivel
\end{verbatim}
\vspace{1mm}
\end{minipage} \\
\hline
\end{tabular}
\end{table}
