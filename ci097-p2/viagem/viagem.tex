Está chegando a grande final do Campeonato Nlogonense de Surf Aquático, que este
ano ocorrerá na cidade de Bonita Horeleninha (BH)! Você decidiu que
irá viajar de sua cidade natal para BH para acompanhar a final.

Existem $N$ cidades em Nlogônia, numeradas de $1$ a $N$. Considere que cidade
$1$ é sua cidade natal, e a cidade $N$ é BH.

Além disso, existem $M$ trechos pelos quais
é possível viajar. Cada trecho pode ser usado para ir de uma cidade para
alguma outra do país. Alguns trechos são feitos de ônibus, enquanto os demais são
feitos de avião. Para cada trecho, você conhece o preço, em reais, da passagem
que deve pagar para poder utilizá-lo.

Para não tornar sua viagem muito cansativa com deslocamentos entre
rodoviárias e aeroportos, você decidiu que irá utilizar \textit{apenas um} meio
de transporte em \textit{toda} sua viagem, isto é, você quer ir para BH
ou utilizando apenas ônibus, ou utilizando apenas aviões.

Sua tarefa é determinar o custo mínimo necessário, em reais, para viajar da sua
cidade natal para BH, dada a restrição que o meio de transporte não deve ser
alterado durante a viagem.

\subsection*{Entrada}

A primeira linha contém dois inteiros $N$ e $M$ ($2 \leq N \leq 100, 1 \leq M
\leq 2(N^2 - N)$), o número de cidades e de trechos, respectivamente. As próximas
$M$ linhas descreve um trecho cada. Cada linha contém quatro inteiros
$A$ $B$ $T$ $R$ ($1 \leq A, B \leq N, A \neq B, T=0$ ou $1$, $1 \leq R \leq 10^4$),
indicando um trecho que sai da cidade $A$ e chega na cidade $B$ (nesta ordem),
feito por ônibus se $T=0$ ou por avião se $T=1$, e cuja passagem custa $R$
reais.

É garantido que existe ao menos um caminho de sua cidade para BH utilizando
apenas um meio de transporte. Além disso, para cada par ordenado de cidades
$(A,B)$, existe no máximo um trecho de $A$ para $B$ para cada meio de transporte
possível (mas note que pode haver um trecho de ônibus e outro de
avião de $A$ para $B$).

\subsection*{Saída}

Imprima uma única linha contendo um inteiro indicando o custo mínimo necessário
para fazer sua viagem, dadas as restrições acima.

\begin{table}[!h]
\centering
\begin{tabular}{|l|l|}
\hline
\begin{minipage}[t]{3in}
\textbf{Exemplo de entrada}
\begin{verbatim}
5 6
1 2 0 200
1 3 1 400
2 4 0 300
3 4 1 300
2 5 0 700
4 5 1 100
\end{verbatim}
\vspace{1mm}
\end{minipage}
&

\begin{minipage}[t]{3in}
\textbf{Exemplo de saída}
\begin{verbatim}
800
\end{verbatim}
\vspace{1mm}
\end{minipage} \\
\hline
\end{tabular}
\end{table}
