\documentclass[12pt,oneside]{article} % Uma Coluna e lingua portuguesa
%\usepackage[T1]{fontenc}        % Permite digitar os acentos de forma normal
\usepackage[utf8]{inputenc}
%\usepackage[english]{babel}
\usepackage[portuges,brazil]{babel}
%\usepackage[latin1]{inputenc}
\usepackage[dvips]{graphicx}    % Permite Gráficos
%\usepackage{times}    % Fonte Times
\usepackage{fancyhdr}
\usepackage{array}
\usepackage{multicol}
\usepackage[colorlinks=true,linkcolor=blue,urlcolor=blue]{hyperref}
\usepackage{nomencl}    % glossario
\usepackage{amssymb}
\usepackage{amsmath}
\usepackage[compact]{titlesec}
\usepackage{wrapfig}

%=======================================================================

% Hifenização das palavras desconhecidas pelo LaTeX
%\hyphenation{}
\paperheight    297mm
\paperwidth     210mm
\voffset         -15mm
\headheight      15pt %% tamanho de letra
\headsep         5mm  %% para o início do texto
\oddsidemargin  -3.0mm
\evensidemargin -3.0mm
\textwidth      167.0mm
\topmargin      005.0mm
\textheight     240.0mm
\footskip       10.0mm

\title{CI097 - Prova 1}

\author{Maratona de Programação}
\date{19 de Outubro de 2016}
\usepackage{indentfirst}
\usepackage{subfig}

\parindent=0pt
\setlength{\parskip}{7pt plus 1pt minus 2pt}
\titlespacing{\section}{0pt}{*0}{*0}
\titlespacing{\subsection}{0pt}{*0}{*0}
\titlespacing{\subsubsection}{0pt}{*0}{*0}

\begin{document}

\begin{center}
\textbf{\Huge CI097 - Prova 2}\\
\vspace{0.2cm}
\textit{19 de Outubro de 2016} \\
\vspace{1.0cm}
\textbf{Servidor BOCA:} \\
\texttt{\large http://maratona.c3sl.ufpr.br/} \\
\vspace{1.0cm}
\end{center}

\pagestyle{fancy}
\renewcommand{\footrulewidth}{0.7pt}
\renewcommand{\headrulewidth}{0.7pt}
\lhead{CI097 - Prova 2}
%\chead{Maratonas de Programação}
\rhead{19 de Outubro de 2016}
\cfoot{\thepage}


% Espaco para o create-zips.sh nao achar
  \section*{Instruções Importantes}

\begin{itemize}

    \item A prova é composta por 5 problemas que valem 20 pontos cada, e um
    problema bônus que também vale 20 pontos.

    \item O problema F é um problema bônus, usado em uma seletiva mas que não foi resolvido.
    Este problema só deve ser considerado caso todos os demais forem resolvidos.

    \item Se sua solução para um problema for aceita, a nota integral
    do problema é garantida.

    \item Se sua solução não for aceita, o professor irá avaliar sua última (mas
            \textbf{apenas} a última) submissão para o problema e atribuir a ela uma
    nota parcial, considerando a corretude do algoritmo utilizado, a
    complexidade computacional do algoritmo, o uso correto de funções e estruturas de dados
    disponíveis utilizadas, e eventuais detalhes da solução.

    \item Em cada problema, cada arquivo de entrada contém \textit{apenas um} caso de
    teste. Sua solução será executada com vários arquivos de entrada
    separadamente.

    \item Sua solução será compilada ou executada com a seguinte linha de comando:
    \begin{itemize}
        \item C: \verb|gcc -static -O2 -lm|
        \item C++: \verb|g++ -static -O2 -lm|
        \item C++11: \verb|g++ -std=c++11 -static -O2 -lm|
        \item Java: \verb|javac|
        \item Pascal: \verb|fpc -Xt -XS -O2|
        \item Python: \verb|python3|
    \end{itemize}

    \item O tempo limite para a execução de sua solução é de 1 segundo para
    todos os seis problemas da prova.

    \item Todas as linhas, tanto na entrada quanto na saída, terminam com o
    caractere de fim-de-linha ($\backslash n$), mesmo quando houver apenas uma única
    linha no arquivo.

    \item Para submissões em \textbf{JAVA}, a classe deverá ter o mesmo nome que
    o \textit{basename} do problema (leia a linha entre o título e o texto do
    problema).

\end{itemize}

\newpage

\clearpage
\section*{A: Viagem para BH} %tle=1
\vspace{-0.52cm}
\noindent \begin{verbatim}Arquivo: viagem.[c|cpp|java|pas|py]\end{verbatim}
Está chegando a grande final do Campeonato Nlogonense de Surf Aquático, que este
ano ocorrerá na cidade de Bonita Horeleninha (BH)! Você decidiu que
irá viajar de sua cidade natal para BH para acompanhar a final.

Existem $N$ cidades em Nlogônia, numeradas de $1$ a $N$. Considere que cidade
$1$ é sua cidade natal, e a cidade $N$ é BH.

Além disso, existem $M$ trechos pelos quais
é possível viajar. Cada trecho pode ser usado para ir de uma cidade para
alguma outra do país. Alguns trechos são feitos de ônibus, enquanto os demais são
feitos de avião. Para cada trecho, você conhece o preço, em reais, da passagem
que deve pagar para poder utilizá-lo.

Para não tornar sua viagem muito cansativa com deslocamentos entre
rodoviárias e aeroportos, você decidiu que irá utilizar \textit{apenas um} meio
de transporte em \textit{toda} sua viagem, isto é, você quer ir para BH
ou utilizando apenas ônibus, ou utilizando apenas aviões.

Sua tarefa é determinar o custo mínimo necessário, em reais, para viajar da sua
cidade natal para BH, dada a restrição que o meio de transporte não deve ser
alterado durante a viagem.

\subsection*{Entrada}

A primeira linha contém dois inteiros $N$ e $M$ ($2 \leq N \leq 100, 1 \leq M
\leq 2(N^2 - N)$), o número de cidades e de trechos, respectivamente. As próximas
$M$ linhas descreve um trecho cada. Cada linha contém quatro inteiros
$A$ $B$ $T$ $R$ ($1 \leq A, B \leq N, A \neq B, T=0$ ou $1$, $1 \leq R \leq 10^4$),
indicando um trecho que sai da cidade $A$ e chega na cidade $B$ (nesta ordem),
feito por ônibus se $T=0$ ou por avião se $T=1$, e cuja passagem custa $R$
reais.

É garantido que existe ao menos um caminho de sua cidade para BH utilizando
apenas um meio de transporte. Além disso, para cada par ordenado de cidades
$(A,B)$, existe no máximo um trecho de $A$ para $B$ para cada meio de transporte
possível (mas note que pode haver um trecho de ônibus e outro de
avião de $A$ para $B$).

\subsection*{Saída}

Imprima uma única linha contendo um inteiro indicando o custo mínimo necessário
para fazer sua viagem, dadas as restrições acima.

\begin{table}[!h]
\centering
\begin{tabular}{|l|l|}
\hline
\begin{minipage}[t]{3in}
\textbf{Exemplo de entrada}
\begin{verbatim}
5 6
1 2 0 200
1 3 1 400
2 4 0 300
3 4 1 300
2 5 0 700
4 5 1 100
\end{verbatim}
\vspace{1mm}
\end{minipage}
&

\begin{minipage}[t]{3in}
\textbf{Exemplo de saída}
\begin{verbatim}
800
\end{verbatim}
\vspace{1mm}
\end{minipage} \\
\hline
\end{tabular}
\end{table}


\clearpage
\section*{B: Pizza antes de BH} %tle=1
\vspace{-0.52cm}
\noindent \begin{verbatim}Arquivo: pizza.[c|cpp|java|pas|py]\end{verbatim}
Está chegando a grande final do Campeonato Nlogonense de Surf Aquático, que este
ano ocorrerá na cidade de Bonita Horeleninha (BH)!
Antes de viajar para BH, você e seus $N-1$ amigos decidiram combinar algum dia
para ir a uma pizzaria, para relaxar e descontrair (e, naturalmente, comer!).

Neste momento está sendo escolhida a data do evento.
Para que todas as peossas possam participar, foi decidido
que o encontro na pizzaria ocorrerá em um data tal que \textit{todas} as $N$
pessoas podem comparecer à pizzaria nesta data. Portanto, nem toda data pode ser
escolhida, pois algumas pessoas podem ter outros compromissos já marcados em
alguns dias.

Dada a lista de datas consideradas para o evento e a informações de quais
pessoas podem comparecer em quais datas, determine se o evento poderá
ocorrer e, em caso positivo, sua data.
Caso mais de uma data seja possível, o evento deve ocorrer o mais cedo
possível.

\subsection*{Entrada}

A primeira linha contém os inteiros $N$ e $D$ ($1 \leq N, D \leq 50$), o número de pessoas e o número de datas consideradas,
respectivamente. As pessoas são numeradas de $1$ a $N$.
As próximas $D$ linhas descrevem uma data considerada. Cada linha começa com a
data na forma $dia/mes/ano$. A linha é seguida de
$N$ inteiros $p_1, p_2,...,p_N$. O
inteiro $p_i$ é 1 se a pessoa $i$ pode comparecer na data considerada, ou $0$
caso contrário.
É garantido que as datas são sempre válidas, e não há zeros à esquerda. Além
disso, as datas são dadas em ordem, do dia mais cedo para o dia mais tarde.

\subsection*{Saída}

Imprima uma linha contendo a data que o evento deve ocorrer, na forma
$dia/mes/ano$, de maneira idêntica à da entrada. Caso não seja
possível realizar o evento, imprima ``\verb|Pizza antes de FdI|'' (sem aspas).

\begin{table}[!h]
\centering
\begin{tabular}{|l|l|}
\hline
\begin{minipage}[t]{3in}
\textbf{Exemplo de entrada}
\begin{verbatim}
4 4
1/6/2016 0 0 1 0
12/7/2016 1 1 1 0
5/10/2016 1 1 1 1
25/12/2016 0 0 0 0
\end{verbatim}
\vspace{1mm}
\end{minipage}
&

\begin{minipage}[t]{3in}
\textbf{Exemplo de saída}
\begin{verbatim}
5/10/2016
\end{verbatim}
\vspace{1mm}
\end{minipage} \\
\hline
\end{tabular}
\end{table}

\begin{table}[!h]
\centering
\begin{tabular}{|l|l|}
\hline
\begin{minipage}[t]{3in}
\textbf{Exemplo de entrada}
\begin{verbatim}
5 3
20/9/2016 0 1 1 1 1
30/9/2016 1 0 1 1 1
1/10/2016 1 1 0 1 1
\end{verbatim}
\vspace{1mm}
\end{minipage}
&

\begin{minipage}[t]{3in}
\textbf{Exemplo de saída}
\begin{verbatim}
Pizza antes de FdI
\end{verbatim}
\vspace{1mm}
\end{minipage} \\
\hline
\end{tabular}
\end{table}


\clearpage
\section*{C: Paodequeijosweeper} %tle=1
\vspace{-0.52cm}
\noindent \begin{verbatim}Arquivo: paodequeijo.[c|cpp|java|pas|py]\end{verbatim}
Está chegando a grande final do Campeonato Nlogonense de Surf Aquático, que este
ano ocorrerá na cidade de Bonita Horeleninha (BH)! Nesta cidade, o jogo
\textit{PãodequeijoSweeper} é bastante popular!

O tabuleiro do jogo consiste em uma matriz de $N$ linhas e $M$ colunas. Cada
célula da matriz contém um pão de queijo ou o número de pães de queijo que
existem nas celulas adjacentes a ela. Uma célula é adjacente a outra se estiver
imediatamente à esquerda, à direita, acima ou abaixo da célula. Note que, se não
contiver um pão de queijo, uma célula deve obrigatoriamente conter um número
entre $0$ e $4$, inclusive.

Dadas as posições dos pães de queijo, determine o tabuleiro do jogo!

\subsection*{Entrada}

A primeira linha contém os inteiros $N$ e $M$ ($1 \leq N, M \leq 100$). As
próximas $N$ linhas contém $M$ inteiros cada, separados por espaços, descrevendo
os pães de queijo no tabuleiro. O $j-$ésimo inteiro da $i-$ésima linha é $1$ se
existe um pão de queijo na linha $i$ e coluna $j$ do tabuleiro, ou $0$ caso
contrário.

\subsection*{Saída}

Imprima $N$ linhas com $M$ inteiros cada, \textit{não separados por espaços},
descrevendo a configuração do tabuleiro. Se uma posição contém um pão de queijo,
imprima \verb|9| para ela; caso contrário, imprima o número cuja posição deve
conter.

\begin{table}[!h]
\centering
\begin{tabular}{|l|l|}
\hline
\begin{minipage}[t]{3in}
\textbf{Exemplo de entrada}
\begin{verbatim}
4 4
0 0 1 1
0 1 0 1
0 0 1 0
1 1 0 1
\end{verbatim}
\vspace{1mm}
\end{minipage}
&

\begin{minipage}[t]{3in}
\textbf{Exemplo de saída}
\begin{verbatim}
0299
1949
1393
9939
\end{verbatim}
\vspace{1mm}
\end{minipage} \\
\hline
\end{tabular}
\end{table}

\begin{table}[!h]
\centering
\begin{tabular}{|l|l|}
\hline
\begin{minipage}[t]{3in}
\textbf{Exemplo de entrada}
\begin{verbatim}
1 2
0 1
\end{verbatim}
\vspace{1mm}
\end{minipage}
&

\begin{minipage}[t]{3in}
\textbf{Exemplo de saída}
\begin{verbatim}
19
\end{verbatim}
\vspace{1mm}
\end{minipage} \\
\hline
\end{tabular}
\end{table}


\clearpage
\section*{D: Passeio em FdI} %tle=1
\vspace{-0.52cm}
\noindent \begin{verbatim}Arquivo: passeio.[c|cpp|java|pas|py]\end{verbatim}
Está chegando a grande final do Campeonato Nlogonense de Surf Aquático.
Ano que vem, a final ocorrerá na cidade de Foça do Iguachim (FdI)!
A cidade é famosa por conter o Parque Nacional do Iguachim, que conta com várias
atrações. Dentre elas, destacam-se as Cataratas do Iguachim, um dos pontos turisticos
mais famosos de Nlogonia!

O Parque conta com $N$ atrações, numeradas de $1$ a $N$. As atrações são
dispostas em uma linha reta no Parque. Desta forma, o Parque pode ser
descrito como uma rua contendo entradas para as atrações $1$, $2$, ..., $N$, onde a
atração $1$ é a mais próxima da entrada do Parque, enquanto a atração $N$
é a mais próxima da saída do Parque. Para não tumultoar o Parque, é
exigido que as atrações sejam visitadas \textit{em ordem} da entrada para a
saída, isto é, se você visitar a atração $i$, você não pode voltar e visitar as
atrações $1$, $2$, ..., $i-1$.

Além disso, existem dois tipos de \textit{tickets} no Parque: os
\textit{tickets} verdes e os \textit{tickets} amarelos. Cada uma das $N$
atrações exigem, como pagamento por sua entrada, uma certa quantidade de
\textit{tickets} de exatamente um tipo. Ao entrar em uma atração, o Parque pode
lhe presentear com uma certa quantidade de \textit{tickets} do outro tipo, isto
é, uma atração que cobra \textit{tickets} verdes como entrada pode lhe dar
\textit{tickets} amarelos como presente, ou vice-versa. Você pode não poder
entrar em uma atração se não tiver \textit{tickets} suficientes para ela, mas
também pode optar não entrar nela se quiser, mesmo se tiver \textit{tickets}
suficientes.

Entretanto, você quer aproveitar o Parque o máximo possível! Dada a quantidade
inicial de \textit{tickets} de cada tipo que você possui e a descrição das atrações do
Parque, determine o número máximo de atrações que podem ser visitadas.

\subsection*{Entrada}

A primeira linha contém o inteiro $N$ ($1 \leq N \leq 40$). A segunda linha
contém dois inteiros $V$ e $A$ ($0 \leq V, A \leq 20$), o número de
\textit{tickets} verdes e amarelos que você possui inicialmente. As próximas $N$
linhas descrevem as atrações do Parque, na ordem da entrada para a saída do
Parque.
Cada linha contém dois inteiros $V_i$ e $A_i$ ($-20 \leq V_i, A_i \leq 20$, $V_i
\times A_i < 0$). Se $V_i < 0$, a atração cobra $|V_i|$ \textit{tickets}
verdes como entrada, e, se visitada, lhe presenteia com $A_i$ \textit{tickets}
amarelos. Caso contrário, ela cobra $|A_i|$ \textit{tickets} amarelos, e, se
visitada, lhe presenteia com $V_i$ \textit{tickets} verdes.

\subsection*{Saída}

Imprima uma linha contendo a quantidade máxima de atrações que podem ser
visitadas.

\begin{table}[!h]
\centering
\begin{tabular}{|l|l|}
\hline
\begin{minipage}[t]{3in}
\textbf{Exemplo de entrada}
\begin{verbatim}
3
10 0
-10 5
-5 4
20 -5
\end{verbatim}
\vspace{1mm}
\end{minipage}
&

\begin{minipage}[t]{3in}
\textbf{Exemplo de saída}
\begin{verbatim}
2
\end{verbatim}
\vspace{1mm}
\end{minipage} \\
\hline
\end{tabular}
\end{table}


\clearpage
\section*{E: Compras em FdI} %tle=1
\vspace{-0.52cm}
\noindent \begin{verbatim}Arquivo: compras.[c|cpp|java|pas|py]\end{verbatim}
Está chegando a grande final do Campeonato Nlogonense de Surf Aquático.
Ano que vem, a final ocorrerá na cidade de Foça do Iguachim (FdI)!
A região de FdI e das cidades próximas é famosa por seu comércio, composto por diversas lojas
que costumam vender diversos produtos a preços mais atraentes que no restante do
país. Você quer aproveitar a viagem para FdI para comprar o novo celular
\textit{Aifôni (R)}!\footnote{Na verdade, você queria um \textit{Sãosunga (R)},
mas este celular é um verdadeiro estouro!}

Existem $N$ lojas na região, numeradas de $1$ a $N$. Todas as lojas vendem o
celular, embora o preço do aparelho pode ser diferente em cada loja. Para não
tornar sua viagem cansativa, você pode considerar não visitar todas as $N$
lojas, mas sim visitar apenas as lojas entre duas dadas lojas $i$ e $j$, inclusive.
Você está interessado na \textit{maior diferença de preços} do aparelho entre as
lojas visitadas. A diferença é dada por $|M - m|$, onde $M$ é o maior preço
dentre as lojas visitadas, e $m$ é o menor.

Além disso, as lojas podem alterar o preço do celular como desejarem! Sua tarefa
é determinar, \textit{para várias consultas}, a maior diferença de preços nas
lojas entre duas dadas lojas, considerando também eventuais alterações de preços nas lojas.

\subsection*{Entrada}

A primeira linha contém o inteiro $N$ ($1 \leq N \leq 10^5$). A segunda linha
contém $N$ inteiros $p_1, p_2, ..., p_N$ ($1 \leq p_i \leq 10^5$). O inteiro
$p_i$ indica o preço inicial do celular na loja $i$.
A terceira linha contém um inteiro $Q$ ($1 \leq Q \leq 10^5$), o número de
operações. As próximas $Q$ linhas descrevem uma operação cada. Cada operação
pode ser descrita de duas formas:

\begin{itemize}
    \item $1$ $i$ $p$ ($1 \leq i \leq N, 1 \leq p \leq 10^5$),
    indicando que o preço do celular foi alterado para $p$ na loja $i$.
    \item $2$ $i$ $j$ ($1 \leq i \leq j \leq N$), indicando uma consulta.
\end{itemize}

\subsection*{Saída}

Para cada consulta, imprima uma linha contendo a maior diferença de preços das
lojas entre as lojas $i$ e $j$, inclusive.

\begin{table}[!h]
\centering
\begin{tabular}{|l|l|}
\hline
\begin{minipage}[t]{3in}
\textbf{Exemplo de entrada}
\begin{verbatim}
4
100 150 90 170
3
2 1 3
1 2 50
2 2 4
\end{verbatim}
\vspace{1mm}
\end{minipage}
&

\begin{minipage}[t]{3in}
\textbf{Exemplo de saída}
\begin{verbatim}
60
120
\end{verbatim}
\vspace{1mm}
\end{minipage} \\
\hline
\end{tabular}
\end{table}


\clearpage
\section*{F: Desafio das Moedas Prateadas} %tle=1
\vspace{-0.52cm}
\noindent \begin{verbatim}Arquivo: desafio.[c|cpp|java|pas|py]\end{verbatim}
O \textit{Desafio das Moedas Prateadas} é um esporte individual popular no reino de Diddykongolândia. O campo e as regras do jogo são descritos a seguir.

O campo do jogo consiste em \textit{locais} e \textit{trechos}. Há $N$ locais no
campo. Um desses locais é o local de \textit{largada}, no qual o jogador inicia o jogo.

Há $M$ trechos no campo. Cada trecho é uma via unidirecional que liga um local do campo a outro local distinto. Os trechos são os únicos meios pelos quais o jogador pode se mover entre os locais do campo. Os trechos do campo são dados de tal forma que:

\begin{itemize}
    \item é possível ir do local de largada a qualquer outro local usando os trechos;
    \item é possível ir de qualquer local do campo ao local de largada usando os trechos;
    \item é impossível sair de um local $l$ e voltar para o mesmo local $l$ sem passar pelo local de largada.
\end{itemize}

Há também $K$ \textit{moedas prateadas} no jogo. Cada moeda está em um local distinto do campo. Se o jogador chegar a um local que contém uma moeda, o jogador pode coletá-la. Não há moeda no local de largada.

O jogador começa o jogo no local de largada. Uma \textit{volta} é completada pelo jogador quando ele retorna ao local de largada após passar por outro(s) local(is).

O jogador deve completar exatamente três voltas. Se o jogador conseguir coletar todas as moedas de prata antes de completar a última volta, ele vence. Caso contrário, ele perde.

Após analizar o campo de jogo, você descobriu quanto tempo leva para atravessar cada trecho. Agora, você deve descobrir se é possível vencer no campo dado e, em caso positivo, qual o tempo mínimo que você deve levar para vencer. Considere instantâneo o tempo para coletar uma moeda e para atravessar um local (sair de um trecho e entrar em outro adjacente).


\subsection*{Entrada}


A entrada inicia com uma linha contendo três inteiros $N$, $M$ e $K$ ($2 \leq N
        \leq 1000, 2 \leq M \leq (N^2 + N - 2)/2, 1 \leq K \leq min\{12, N-1\}$), indicando o número de locais, de trechos e de moedas no campo. Os locais são numerados de 1 a $N$. O local 1 é o local de largada.

As próximas $M$ linhas descrevem os trechos. Cada trecho é descrito por três
inteiros $l_a$, $l_b$ e $t$ ($1 \leq l_a, l_b \leq N, l_a \neq l_b, 1 \leq t
        \leq 10^4$), indicando que há um trecho que leva do local $l_a$ para o
local $l_b$ que é atravessado em $t$ segundos.

A última linha contém $K$ inteiros distintos $k_i$ ($2 \leq k_i \leq N$ para $1 \leq i \leq K$) indicando os locais das moedas prateadas.

\subsection*{Saída}

Se não é possível vencer o jogo, imprima uma linha contendo
``\verb|impossivel|'' (sem aspas). Caso
contrário, imprima uma linha contendo o menor tempo necessário, em segundos, para
vencer o jogo.

\clearpage
\begin{table}[!h]
\centering
\begin{tabular}{|l|l|}
\hline
\begin{minipage}[t]{3in}
\textbf{Exemplo de entrada}
\begin{verbatim}
4 6 3
1 2 1
1 3 1
1 4 1
2 1 1
3 1 1
4 1 1
2 3 4
\end{verbatim}
\vspace{1mm}
\end{minipage}
&

\begin{minipage}[t]{3in}
\textbf{Exemplo de saída}
\begin{verbatim}
6
\end{verbatim}
\vspace{1mm}
\end{minipage} \\
\hline
\end{tabular}
\end{table}

\begin{table}[!h]
\centering
\begin{tabular}{|l|l|}
\hline
\begin{minipage}[t]{3in}
\textbf{Exemplo de entrada}
\begin{verbatim}
5 7 2
1 2 1
1 3 2
2 3 3
3 4 7
3 5 3
4 5 2
5 1 4
2 4
\end{verbatim}
\vspace{1mm}
\end{minipage}
&

\begin{minipage}[t]{3in}
\textbf{Exemplo de saída}
\begin{verbatim}
35
\end{verbatim}
\vspace{1mm}
\end{minipage} \\
\hline
\end{tabular}
\end{table}

\begin{table}[!h]
\centering
\begin{tabular}{|l|l|}
\hline
\begin{minipage}[t]{3in}
\textbf{Exemplo de entrada}
\begin{verbatim}
5 8 4
1 2 3
1 3 2
1 4 10
1 5 7
2 1 5
3 1 3
4 1 1
5 1 12
4 5 3 2
\end{verbatim}
\vspace{1mm}
\end{minipage}
&

\begin{minipage}[t]{3in}
\textbf{Exemplo de saída}
\begin{verbatim}
impossivel
\end{verbatim}
\vspace{1mm}
\end{minipage} \\
\hline
\end{tabular}
\end{table}


\end{document}
