Está chegando a grande final do Campeonato Nlogonense de Surf Aquático, que este
ano ocorrerá na cidade de Bonita Horeleninha (BH)!
Antes de viajar para BH, você e seus $N-1$ amigos decidiram combinar algum dia
para ir a uma pizzaria, para relaxar e descontrair (e, naturalmente, comer!).

Neste momento está sendo escolhida a data do evento.
Para que todas as peossas possam participar, foi decidido
que o encontro na pizzaria ocorrerá em um data tal que \textit{todas} as $N$
pessoas podem comparecer à pizzaria nesta data. Portanto, nem toda data pode ser
escolhida, pois algumas pessoas podem ter outros compromissos já marcados em
alguns dias.

Dada a lista de datas consideradas para o evento e a informações de quais
pessoas podem comparecer em quais datas, determine se o evento poderá
ocorrer e, em caso positivo, sua data.
Caso mais de uma data seja possível, o evento deve ocorrer o mais cedo
possível.

\subsection*{Entrada}

A primeira linha contém os inteiros $N$ e $D$ ($1 \leq N, D \leq 50$), o número de pessoas e o número de datas consideradas,
respectivamente. As pessoas são numeradas de $1$ a $N$.
As próximas $D$ linhas descrevem uma data considerada. Cada linha começa com a
data na forma $dia/mes/ano$. A linha é seguida de
$N$ inteiros $p_1, p_2,...,p_N$. O
inteiro $p_i$ é 1 se a pessoa $i$ pode comparecer na data considerada, ou $0$
caso contrário.
É garantido que as datas são sempre válidas, e não há zeros à esquerda. Além
disso, as datas são dadas em ordem, do dia mais cedo para o dia mais tarde.

\subsection*{Saída}

Imprima uma linha contendo a data que o evento deve ocorrer, na forma
$dia/mes/ano$, de maneira idêntica à da entrada. Caso não seja
possível realizar o evento, imprima ``\verb|Pizza antes de FdI|'' (sem aspas).

\begin{table}[!h]
\centering
\begin{tabular}{|l|l|}
\hline
\begin{minipage}[t]{3in}
\textbf{Exemplo de entrada}
\begin{verbatim}
4 4
1/6/2016 0 0 1 0
12/7/2016 1 1 1 0
5/10/2016 1 1 1 1
25/12/2016 0 0 0 0
\end{verbatim}
\vspace{1mm}
\end{minipage}
&

\begin{minipage}[t]{3in}
\textbf{Exemplo de saída}
\begin{verbatim}
5/10/2016
\end{verbatim}
\vspace{1mm}
\end{minipage} \\
\hline
\end{tabular}
\end{table}

\begin{table}[!h]
\centering
\begin{tabular}{|l|l|}
\hline
\begin{minipage}[t]{3in}
\textbf{Exemplo de entrada}
\begin{verbatim}
5 3
20/9/2016 0 1 1 1 1
30/9/2016 1 0 1 1 1
1/10/2016 1 1 0 1 1
\end{verbatim}
\vspace{1mm}
\end{minipage}
&

\begin{minipage}[t]{3in}
\textbf{Exemplo de saída}
\begin{verbatim}
Pizza antes de FdI
\end{verbatim}
\vspace{1mm}
\end{minipage} \\
\hline
\end{tabular}
\end{table}
