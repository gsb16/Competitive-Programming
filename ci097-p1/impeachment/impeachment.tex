\textit{Analógimôn Go} é um jogo bastante popular.
Os jogadores de \textit{Analógimôn Go} são divididos em três grandes
times: Time \textit{Valor}, Time \textit{Instinto} e Time \textit{Místico}, que
são liderados pelos seus líderes Kandera, Esparky e Blanque, respectivamente.
Naturalmente, você faz parte de um desses times!

O líder do seu time está sendo acusado de infringir as regras do jogo por
gerenciar incorretamente os doces recebidos do Professor que são destinados ao
time. Isto criou uma grande polêmica dentro da equipe: alguns jogadores defendem
que o líder realmente agiu incorretamente e deve sofrer um \textit{impeachment}
e ser afastado de seu cargo, enquanto outros defendem que ele não infringiu as
regras, que a acusação é inverídica e que ele deve continuar no cargo.

Para resolver a situação, uma votação será realizada entre todos os $N$
jogadores do seu time. Cada jogador deverá votar se o \textit{impeachment} deve
ou não ocorrer. Se o número de votos favoráveis ao \textit{impeachment} foi
maior ou igual a 2/3 (dois terços) do total de jogadores, o líder será afastado.
Caso contrário, a acusação é arquivada e ele continuará no cargo.

Dados os votos de todos os jogadores, determine o resultado da votação.

\subsection*{Entrada}

A primeira linha contém o inteiro $N$ ($1 \leq N \leq 10^5$), o número de
jogadores em seu time.
A próxima linha contém $N$ inteiros $v_1$, ..., $v_N$ ($v_i = 0$ ou $1$),
indicando os votos dos jogadores. O valor $1$ indica um voto favorável ao
\textit{impeachment}, enquanto o valor $0$ indica um voto contrário ao mesmo.

\subsection*{Saída}

Imprima uma linha contendo a palavra \verb|impeachment| se o líder deve
ser afastado de seu cargo, ou \verb|acusacao arquivada| caso contrário.

\begin{table}[!h]
\centering
\begin{tabular}{|l|l|}
\hline
\begin{minipage}[t]{3in}
\textbf{Exemplo de entrada}
\begin{verbatim}
6
1 0 1 1 0 1
\end{verbatim}
\vspace{1mm}
\end{minipage}
&

\begin{minipage}[t]{3in}
\textbf{Exemplo de saída}
\begin{verbatim}
impeachment
\end{verbatim}
\vspace{1mm}
\end{minipage} \\
\hline
\end{tabular}
\end{table}

\begin{table}[!h]
\centering
\begin{tabular}{|l|l|}
\hline
\begin{minipage}[t]{3in}
\textbf{Exemplo de entrada}
\begin{verbatim}
5
0 1 1 1 0
\end{verbatim}
\vspace{1mm}
\end{minipage}
&

\begin{minipage}[t]{3in}
\textbf{Exemplo de saída}
\begin{verbatim}
acusacao arquivada
\end{verbatim}
\vspace{1mm}
\end{minipage} \\
\hline
\end{tabular}
\end{table}
