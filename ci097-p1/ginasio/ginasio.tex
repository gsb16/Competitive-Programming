\textit{Analógimôn Go} é um jogo bastante popular. Em sua jornada, o jogador percorre diversas
cidades capturando pequenos monstrinhos virtuais, chamados \textit{analógimôns}.
As cidades contém localidades especiais chamadas de
\textit{ginásios}. Ao chegar a um ginásio, um jogador pode \textit{tentar}
colocar um de seus analógimôns dentro dele.

Cada anlógimôn tem dois inteiros associados a ele: seu \textit{Poder de Combate}
(PC) e seu \textit{Número de Ataques} (NA). Além disso, um ginásio tem associado
a ele um \textit{Intervalo de Poder} (IP). Ao tentar colocar um
analógimôn em um ginásio, o jogo verifica quantos são os analógimôns já
presentes no ginásio cuja diferença do seu PC para o PC do analógimôn sendo
colocado é de no máximo IP. Se esta quantidade for menor ou igual
ao NA do analógimôn sendo colocado, o monstrinho é inserido no ginásio com
sucesso. Caso contrário, ele não é colocado no ginásio.
Em ambos os casos, os analógimôns que já estavam no ginásio
continuam no ginásio.
Como exemplo, considere um ginásio com IP=$3$ com analógimôns de PC iguais a
$5$, $8$,
$13$ e $20$. Se um jogador tenta colocar um analógimôn de PC=$10$ e NA=$4$, o jogo
contará quantos analógimôns há no ginásio com PC entre $10-3 = 7$ e $10+3 = 13$,
inclusive. Como há dois analógimôns neste caso, o monstrinho é colocado com
sucesso no ginásio, pois $2 \leq 4$. O ginásio passa a conter analógimôns de PC
iguais a $5$, $8$, $10$, $13$ e $20$.

Dadas as informações sobre um ginásio e as tentativas de colocar analógimôns
dentro dele, determine quantos analógimôns ficarão no ginásio após todas as
tentativas. Considere que o ginásio inicialmente não contém nenhum analógimôn.

\subsection*{Entrada}

A primeira linha contém os inteiros $IP$ e $M$ ($1 \leq IP, M \leq 10^5$), o
IP do ginásio e o número de tentativas, respectivamente.
As próximas $M$ linhas descrevem as tentativas de colocar um analógimôn no
ginásio, na ordem em que são feitas. Cada linha contém dois inteiros $PC$ e $NA$
($1 \leq PC, NA \leq 10^5$), indicando o PC e o NA
do analógimôn, respectivamente. O Poder de Combate de todos analógimôns são
distintos.

\subsection*{Saída}

Imprima uma linha com um inteiro indicando quantos analógimôns
ficarão no ginásio.

\begin{table}[!h]
\centering
\begin{tabular}{|l|l|}
\hline
\begin{minipage}[t]{3in}
\textbf{Exemplo de entrada}
\begin{verbatim}
3 7
5 2
13 1
8 1
20 5
6 1
11 1
10 4
\end{verbatim}
\vspace{1mm}
\end{minipage}
&

\begin{minipage}[t]{3in}
\textbf{Exemplo de saída}
\begin{verbatim}
5
\end{verbatim}
\vspace{1mm}
\end{minipage} \\
\hline
\end{tabular}
\end{table}
