\textit{Analógimôn Go} é um jogo bastante popular. Em sua jornada, o jogador percorre diversas
cidades capturando pequenos monstrinhos virtuais, chamados \textit{analógimôns}.

Você é um experiente jogador e já capturou $N$ analógimôns, numerados de $1$ a
$N$. Você capturou tantos monstrinhos que já está difícil levar todos com você
em sua jornada. Por isso, você pode se livrar de alguns de seus
monstrinhos transferindo-os para o Professor.

Ao transferir o analógimôn $i$ (para $1 \leq i \leq N$) para o Professor, você ganha
$D_i$ doces do Professor em troca do monstrinho. Como os doces são itens muito
importantes no jogo, você
quer transferir quais e quantos analógimôns forem necessários para ter a maior
quantidade possível de doces!

Entretanto, o analógimôn $i$ (para $1 \leq i \leq N$) pesa $P_i$ kg, e, devido a uma limitação de
espaço no laboratório do Professor, ele não pode receber analógimôns cuja soma
total dos pesos é maior que $K$ kg.

Sua tarefa é determinar a quantidade máxima de doces que você pode obter
transferindo seus monstrinhos, respeitando a limitação de espaço do laboratório do Professor.


\subsection*{Entrada}

A primeira linha contém os inteiros $N$ e $K$
($1 \leq N \leq 100, 1 \leq K \leq 10^4$), o número de analógimôns que você
capturou e a
capacidade do laboratório do Professor, em kg, respectivamente.
A segunda linha contém $N$ inteiros $D_1$, ..., $D_N$ ($1 \leq D_i \leq 10^4$), indicando quantos doces
você ganhará pela transferência de cada analógimôn. A terceira linha contém $N$
inteiros $P_1$, ..., $P_N$ ($1 \leq P_i \leq 10^4$) indicando o peso de cada analógimôn, em kg.

\subsection*{Saída}

Imprima uma única linha contendo a quantidade máxima de doces que você pode
obter.

\begin{table}[!h]
\centering
\begin{tabular}{|l|l|}
\hline
\begin{minipage}[t]{3in}
\textbf{Exemplo de entrada}
\begin{verbatim}
4 52
1 8 14 22
4 12 20 30
\end{verbatim}
\vspace{1mm}
\end{minipage}
&

\begin{minipage}[t]{3in}
\textbf{Exemplo de saída}
\begin{verbatim}
36
\end{verbatim}
\vspace{1mm}
\end{minipage} \\
\hline
\end{tabular}
\end{table}

\begin{table}[!h]
\centering
\begin{tabular}{|l|l|}
\hline
\begin{minipage}[t]{3in}
\textbf{Exemplo de entrada}
\begin{verbatim}
3 2
9 5 2
12 8 42
\end{verbatim}
\vspace{1mm}
\end{minipage}
&

\begin{minipage}[t]{3in}
\textbf{Exemplo de saída}
\begin{verbatim}
0
\end{verbatim}
\vspace{1mm}
\end{minipage} \\
\hline
\end{tabular}
\end{table}
