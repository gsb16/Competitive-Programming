\textit{Analógimôn Go} é um jogo bastante popular. Em sua jornada, o jogador percorre diversas
cidades capturando pequenos monstrinhos virtuais, chamados \textit{analógimôns}.
Existem várias espécies de analógimôns. Cada espécie é de
(exatamente) um \textit{tipo}, como \textit{fogo}, \textit{água}, \textit{elétrico}, etc.
Algumas espécies podem ser do mesmo tipo, enquanto outras podem se tipos
diferentes.

No manual oficial do jogo consta que algumas espécies são do
mesmo tipo. Entretanto, o manual pode não apresentar esta informação para todos
os pares de espécies que são do mesmo tipo. Por exemplo, se o manual indica
que uma espécie $a$ é do mesmo tipo que uma espécie $b$, e que uma espécie $b$ é
do mesmo tipo que uma espécie $c$, então as espécies $a$ e $c$ certamente são do mesmo
tipo, embora esta informação pode não constar no manual.

Você capturou um analógimôn de uma certa espécie. Sua tarefa é determinar o
menor número possível de espécies que certamente são do mesmo tipo da espécie
do seu analógimôn, de acordo com as informações contidas no manual.

\subsection*{Entrada}

A primeira linha contém dois inteiros $N$ e $M$ ($1 \leq N \leq 1000, 0 \leq M
\leq \frac{N \times (N-1)}{2})$, o número de espécies de analógimôns e o
número de informações presentes no manual, respectivamente. As espécies são
numeradas de $1$ a $N$.
Cada uma das próximas $M$ linhas contém uma informação presente no manual. Cada
linha contém dois inteiros $a$ e $b$ ($1 \leq a, b \leq N, a \neq b$), indicando
que as espécies $a$ e $b$ são do mesmo tipo.
A última linha contém um inteiro $E$ ($1 \leq E \leq N$), indicando a espécie de
seu analógimôn.

\subsection*{Saída}

Imprima uma linha com um inteiro indicando a menor quantidade de espécies de
analógimôns que certamente são do mesmo tipo da espécie do seu analógimôn, de
acordo com o manual. Note que a espécie do seu analógimôn também deve
ser contada.

\begin{table}[!h]
\centering
\begin{tabular}{|l|l|}
\hline
\begin{minipage}[t]{3in}
\textbf{Exemplo de entrada}
\begin{verbatim}
5 3
1 3
3 5
2 4
1
\end{verbatim}
\vspace{1mm}
\end{minipage}
&

\begin{minipage}[t]{3in}
\textbf{Exemplo de saída}
\begin{verbatim}
3
\end{verbatim}
\vspace{1mm}
\end{minipage} \\
\hline
\end{tabular}
\end{table}

\begin{table}[!h]
\centering
\begin{tabular}{|l|l|}
\hline
\begin{minipage}[t]{3in}
\textbf{Exemplo de entrada}
\begin{verbatim}
3 1
1 2
3
\end{verbatim}
\vspace{1mm}
\end{minipage}
&

\begin{minipage}[t]{3in}
\textbf{Exemplo de saída}
\begin{verbatim}
1
\end{verbatim}
\vspace{1mm}
\end{minipage} \\
\hline
\end{tabular}
\end{table}
